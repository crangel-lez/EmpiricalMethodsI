% This LaTeX was auto-generated from MATLAB code.
% To make changes, update the MATLAB code and export to LaTeX again.

\documentclass{article}

\usepackage[utf8]{inputenc}
\usepackage[T1]{fontenc}
\usepackage{lmodern}
\usepackage{graphicx}
\usepackage{color}
\usepackage{listings}
\usepackage{hyperref}
\usepackage{amsmath}
\usepackage{amsfonts}
\usepackage{epstopdf}
\usepackage{matlab}

\sloppy
\epstopdfsetup{outdir=./}
\graphicspath{ {./main_exe_images/} }

\begin{document}

\matlabtitle{Homework \#5}

\matlabheading{Carlos Rangel}

\begin{matlabcode}
%Add CompEcon toolbox to path
addpath(genpath('../compecon2011'))

% Load Homework Data
load hw5.mat

% Store Data
global X;
global Y;
global Z;

X=data.X';
Y=data.Y';
Z=data.Z';
\end{matlabcode}


\matlabheading{Question 1.}

\begin{par}
\begin{flushleft}
Assume $u_i=0$ for all i. (i.e. take $u_i$ out of the model, so that $u_0=\sigma_u=\sigma_{u\beta}=0$). Use Guassian Quadrature using 20 nodes to calculate the log-likelihood function when $\beta_0=0.1, \sigma_\beta=1$, and $\gamma =0$.
\end{flushleft}
\end{par}

\begin{matlabcode}
% Number of Nodes to Use
global n_nodes;
n_nodes=20;

% Compute log-likelihood using Gaussian Quadrature
gq_loglike([0 0.1 1])
\end{matlabcode}
\begin{matlaboutput}
ans = -1.2571e+03
\end{matlaboutput}


\matlabheading{Question 2}

\begin{par}
\begin{flushleft}
Now use Monte Carlo Methods using 100 nodes to calculate the log-likelihood function.
\end{flushleft}
\end{par}

\begin{matlabcode}
n_nodes=100;

% Compute the log-likelihood using Monte-Carlo (Fix Seed at 1).
mc_loglike1([0 0.1 1],1)
\end{matlabcode}
\begin{matlaboutput}
ans = -1.2602e+03
\end{matlaboutput}


\matlabheading{Question 3.}

\begin{par}
\begin{flushleft}
Maximize(or minimize the negative) log-likelihood function with respect to the parameters using both integration techniques above. Use Matlab's fmincon without a supplied derivative to max (min) your objective function.
\end{flushleft}
\end{par}

\begin{par}
\begin{flushleft}
Using First Integration Technique
\end{flushleft}
\end{par}

\begin{matlabcode}
 % Initial Guess
x0=0.5*ones(1,3)
\end{matlabcode}
\begin{matlaboutput}
x0 = 1x3    
    0.5000    0.5000    0.5000

\end{matlaboutput}
\begin{matlabcode}

% Define Negative of Loglikelihood
neg_gqloglike=@(x) -gq_loglike(x);

% Lower Bounds
lb=[-Inf ; -Inf; eps];

% Upper Bounds 
ub=[Inf; Inf; Inf];

% Maximization
[gq_argmax, gq_maxval]=fmincon(neg_gqloglike, x0, [], [], [], [], lb, ub);
\end{matlabcode}
\begin{matlaboutput}
Local minimum possible. Constraints satisfied.

fmincon stopped because the size of the current step is less than
the default value of the step size tolerance and constraints are 
satisfied to within the default value of the constraint tolerance.

<stopping criteria details>
\end{matlaboutput}
\begin{matlabcode}

% Argmax
% First element corresponds to gamma
% Second Element corresponds to beta_0
% Third element corresponds to variance of beta
gq_argmax
\end{matlabcode}
\begin{matlaboutput}
gq_argmax = 1x3    
   -0.5060    2.4937    1.3738

\end{matlaboutput}
\begin{matlabcode}

% Maximized Value
gq_maxval
\end{matlabcode}
\begin{matlaboutput}
gq_maxval = 536.7241
\end{matlaboutput}


\begin{par}
\begin{flushleft}
Using Second Integration Technique
\end{flushleft}
\end{par}

\begin{matlabcode}
% Initial Guess
x0=0.5*ones(1,3)
\end{matlabcode}
\begin{matlaboutput}
x0 = 1x3    
    0.5000    0.5000    0.5000

\end{matlaboutput}
\begin{matlabcode}

% Define Negative of Log-Likelihood (Fix Seed at 1)
neg_mcloglike=@(x) -mc_loglike1(x,1);

% Lower Bounds
lb=[-Inf ; -Inf; eps];

% Upper Bounds
ub=[Inf; Inf; Inf];

% Maximization
[mc_argmax, mc_maxval]=fmincon(neg_mcloglike, x0, [], [], [], [], lb, ub);
\end{matlabcode}
\begin{matlaboutput}
Local minimum possible. Constraints satisfied.

fmincon stopped because the size of the current step is less than
the default value of the step size tolerance and constraints are 
satisfied to within the default value of the constraint tolerance.

<stopping criteria details>
\end{matlaboutput}
\begin{matlabcode}

% Argmax
% First element corresponds to gamma
% Second Element corresponds to beta_0
% Third element corresponds to variance of beta
mc_argmax
\end{matlabcode}
\begin{matlaboutput}
mc_argmax = 1x3    
   -0.5053    2.5524    1.1551

\end{matlaboutput}
\begin{matlabcode}

% Maximized Value
mc_maxval
\end{matlabcode}
\begin{matlaboutput}
mc_maxval = 535.9903
\end{matlaboutput}


\matlabheading{Question 4}

\begin{matlabcode}
% Initial Guess
x0=[0,2,1.5,5,0.5,10]
\end{matlabcode}
\begin{matlaboutput}
x0 = 1x6    
         0    2.0000    1.5000    5.0000    0.5000   10.0000

\end{matlaboutput}
\begin{matlabcode}

% Lower Bounds
lb=[-Inf; -Inf; -Inf; eps; -Inf; eps];

% Upper Bounds
ub=[Inf; Inf; Inf; Inf; Inf; Inf];

% Define Negative of Log Likelihood (Fix Seed at 1)
neg_full_mcloglike=@(x) -full_mcloglike(x,1);

% Maximization
[full_argmax, full_maxval]=fmincon(neg_full_mcloglike, x0, [], [], [], [], lb, ub);
\end{matlabcode}
\begin{matlaboutput}
Local minimum possible. Constraints satisfied.

fmincon stopped because the size of the current step is less than
the default value of the step size tolerance and constraints are 
satisfied to within the default value of the constraint tolerance.

<stopping criteria details>
\end{matlaboutput}
\begin{matlabcode}

% Argmax
% First element corresponds to gamma
% Second Element corresponds to beta_0
% Third element corresponds to  u_0
% Fourth element corresponds to element (1,1) of the Cholesky matrix of sigma2 matrix
% Fifth element corresponds to element (2,1) of the Cholesky matrix
% Sixth element corresponds to element (2,2) of the Cholesky matrix

full_argmax(1:3)
\end{matlabcode}
\begin{matlaboutput}
ans = 1x3    
   -0.6832    3.1612    1.3677

\end{matlaboutput}
\begin{matlabcode}

% Maximized value
full_maxval
\end{matlabcode}
\begin{matlaboutput}
full_maxval = 463.8203
\end{matlaboutput}
\begin{matlabcode}

% Variance-Covariance Matrix

% Cholesky Matrix
R=[full_argmax(4) 0; full_argmax(5) full_argmax(6)];

% Variance-Covariance Matrix
sigma=R*R'
\end{matlabcode}
\begin{matlaboutput}
sigma = 2x2    
    1.6484    0.7016
    0.7016    1.7956

\end{matlaboutput}

\end{document}

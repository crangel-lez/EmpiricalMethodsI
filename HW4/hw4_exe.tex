% This LaTeX was auto-generated from MATLAB code.
% To make changes, update the MATLAB code and export to LaTeX again.

\documentclass{article}

\usepackage[utf8]{inputenc}
\usepackage[T1]{fontenc}
\usepackage{lmodern}
\usepackage{graphicx}
\usepackage{color}
\usepackage{listings}
\usepackage{hyperref}
\usepackage{amsmath}
\usepackage{amsfonts}
\usepackage{epstopdf}
\usepackage{matlab}

\sloppy
\epstopdfsetup{outdir=./}
\graphicspath{ {./hw4_exe_images/} }

\matlabmultipletitles

\begin{document}

\matlabtitle{Homework \#4}

\matlabtitle{Carlos Rangel}

\vspace{1em}

\matlabheading{Question 1}

\begin{par}
\begin{flushleft}
Write an algorithm that uses the dart-throwing method to approximate pi using a quasi-Monte Carlo method
\end{flushleft}
\end{par}

\begin{matlabcode}
% Number of Points
npoints=10000000;

% Draw npoints Neiderrieter points on the unit square:

[points, weights] = qnwequi(npoints, [0 0], [1 1], 'N'); 

% Evaluate the function at these points

fun_evals=4*ind_function(points);

% Average the function evaluations

pi_qmc=mean(fun_evals)
\end{matlabcode}
\begin{matlaboutput}
pi_qmc = 3.1416
\end{matlaboutput}


\matlabheading{Question 2}

\begin{matlabcode}
% Write an algorithm that uses the dart throwing method to approximate pi
% using a Newton-Cotes method

% We are going to use the trapezoid rule

% Number of nodes in the x direction
xnodes=10000;

% Number of nodes in the y direction
ynodes=10000;

% Obtain nodes and weights
[points, weights] = qnwtrap([xnodes ynodes], [0 0], [1 1]);

% Approximate pi
pi_nc=weights'*4*ind_function(points)
\end{matlabcode}
\begin{matlaboutput}
pi_nc = 3.1416
\end{matlaboutput}


\matlabheading{Question 3}

\begin{matlabcode}
% Approximate pi based on the above integral using a quasi-Monte Carlo
% approach

% Draw equidistributed sequence

% Number of Points
npoints=10000000;

% Draw npoints Neiderrieter points on the unit interval:
[points, weights] = qnwequi(npoints, 0, 1, 'N');

% Evaluate the function at these points

fun_evals=integrand(points);

% Average the function evaluations

pi_qmc=mean(fun_evals)
\end{matlabcode}
\begin{matlaboutput}
pi_qmc = 3.1416
\end{matlaboutput}


\matlabheading{Question 4}

\begin{par}
\begin{flushleft}
Write an algorithm that uses the dart-throwing method to approximate pi using a Newton-Cotes approach to computing integrals
\end{flushleft}
\end{par}

\begin{matlabcode}
% We are going to use the trapezoid rule

% Number of nodes in the x direction
xnodes=10000000;

% Obtain nodes and weights
[points, weights] = qnwtrap(xnodes, 0, 1);

% Approximate pi
pi_nc=weights'*integrand(points)
\end{matlabcode}
\begin{matlaboutput}
pi_nc = 3.1416
\end{matlaboutput}


\matlabheading{Question 5}

\begin{par}
\begin{flushleft}
Prepare a table which shows the mean-squared error of 200 simulations of pseudo-MC integration using 100, 1000, and 10,000 draws. Compare this to the squared error of the quasi-MC and Newton-Coates methods for the same number of quadrature draws (i.e. nodes).
\end{flushleft}
\end{par}

\matlabheading{For 100 draws}

\begin{matlabcode}
ndraws=100;

% Number of Simulations
ns=200;

% Vector for Storing integral estimates of each simulation
pi_mc_100=zeros(ns,1);

for i=1:ns
 % Draw 100 pseudo-random numbers
 [nodes, weights] = qnwequi(ndraws, [0 0], [1 1], 'R');
 % Evaluate the pi approximation
 pi_mc_100(i)=weights'*4*ind_function(nodes);
end

% Mean Squared Error
mse_100= mean ( ( pi_mc_100 - pi ).^2 )
\end{matlabcode}
\begin{matlaboutput}
mse_100 = 0.0277
\end{matlaboutput}


\matlabheading{For 1000 draws}

\begin{matlabcode}
ndraws=1000;

% Number of Simulations
ns=200;

% Vector for Storing integral estimates
pi_mc_1000=zeros(ns,1);

for i=1:ns
 % Draw 1000 pseud0-random numbers
 [nodes, weights] = qnwequi(ndraws, [0 0], [1 1], 'R');
 % Evaluate the indicator function at these nodes
 pi_mc_1000(i)=weights'*4*ind_function(nodes);
end

% Mean Squared Error
mse_1000 = mean ( ( pi_mc_1000 - pi ).^2 )
\end{matlabcode}
\begin{matlaboutput}
mse_1000 = 0.0025
\end{matlaboutput}


\matlabheading{For 10,000 draws}

\begin{matlabcode}
ndraws=10000;

% Number of Simulations
ns=200;

% Vector for Storing integral estimates
pi_mc_10000=zeros(ns,1);

for i=1:ns
 % Draw 100 pseudo-random numbers
 [nodes, weights] = qnwequi(ndraws, [0 0], [1 1], 'R');
 % Evaluate the indicator function at these nodes
 pi_mc_10000(i)=weights'*4*ind_function(nodes);
end

% Mean Squared Error
mse_10000= mean ( ( pi_mc_10000 - pi ).^2 )
\end{matlabcode}
\begin{matlaboutput}
mse_10000 = 2.8866e-04
\end{matlaboutput}


\matlabheading{Calculate squared errors for quasi-Monte Carlo and Newton-Coates}

\begin{matlabcode}
% Vector for Storing Squared Errors of quasi-Monte Carlo
sqe_qmc=zeros(3,1);

% First do quasi-Monte Carlo with 100 nodes
nodes=100;

% Starting Position for storing Squared Errors
i=1;

while nodes<=10000
 
 % Draw npoints Neiderrieter points on the unit square:
 [points, weights] = qnwequi(nodes, [0 0], [1 1], 'N');
 
 % Evaluate the function at these points
 fun_evals=4*ind_function(points);
 
 % Obtain pi approximation
 pi_qmc=mean( fun_evals );

 % Calculate the squared error of the current integration approximation
 sqe_qmc(i) = (pi_qmc-pi)^2;
 
 % Increase the number of nodes by an order of magnitude
 nodes=10*nodes;
 
 % Increase counter variable i
 i=i+1;
 
end
\end{matlabcode}


\matlabheading{Squared Errors for Newton-Coates}

\begin{matlabcode}
% Vector for Storing Squared Errors of Newton-Coates
sqe_nc=zeros(3,1);

% First do Newton-Coates with 100
nodes=100;

% Starting position for storing squared errors
i=1;

while nodes<=10000
 
 % Number of nodes in the x direction
 xnodes=nodes;
 % Number of nodes in the y direction
 ynodes=nodes;
 
 % Obtain nodes and weights
 [points, weights] = qnwtrap([xnodes ynodes], [0 0], [1 1]);
 
 % Approximate pi
 pi_nc=weights'*4*ind_function(points);
 
 % Calculate the squared error of the current integration approximation
 sqe_nc(i) = ( pi_nc - pi )^2;
 
 % Increase the number of nodes by an order of magnitude
 nodes=10*nodes;
 
 % Increase counter variable
 i=i+1;
 
end
\end{matlabcode}


\matlabheading{Display Tables comparing Mean Squared Errors}

\begin{matlabcode}
mse_mc=[mse_100; mse_1000; mse_10000];

varNames={'MonteCarlo', 'QuasiMonteCarlo', 'NewtonCoates'};

rowNames={'100', '1000', '10000'};
disp('Mean Squared Error Comparison');
\end{matlabcode}
\begin{matlaboutput}
Mean Squared Error Comparison
\end{matlaboutput}
\begin{matlabcode}
MSE=table(mse_mc, sqe_qmc, sqe_nc, 'VariableNames', varNames, 'RowNames', rowNames)
\end{matlabcode}
\begin{matlabtableoutput}
{
\begin{tabular}{|c|c|c|c|}
\hline
\mlcell{ } & \mlcell{MonteCarlo} & \mlcell{QuasiMonteCarlo} & \mlcell{NewtonCoates} \\ \hline
\mlcell{1 100} & \mlcell{0.0277} & \mlcell{4.6624e-04} & \mlcell{1.3305e-05} \\ \hline
\mlcell{2 1000} & \mlcell{0.0025} & \mlcell{2.5365e-06} & \mlcell{2.4940e-08} \\ \hline
\mlcell{3 10000} & \mlcell{0.0003} & \mlcell{6.2830e-07} & \mlcell{1.1405e-11} \\ 
\hline
\end{tabular}
}
\end{matlabtableoutput}

\end{document}
